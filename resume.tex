% !TEX program = xelatex

\documentclass{resume}
%\usepackage{zh_CN-Adobefonts_external} % Simplified Chinese Support using external fonts (./fonts/zh_CN-Adobe/)
%\usepackage{zh_CN-Adobefonts_internal} % Simplified Chinese Support using system fonts

\begin{document}
\pagenumbering{gobble} % suppress displaying page number

\name{Qianhao ZHANG}

\basicInfo{
  \email{zhangqianhao1408@gmail.com} \textperiodcentered\ 
  \phone{(+1) 510-993-4360} \textperiodcentered\ 
  \linkedin[qianhaoz]{https://www.linkedin.com/in/qianhaoz/} \textperiodcentered\ 
  \github[jasonlovescoding]{https://github.com/jasonlovescoding}}

\section{\faGraduationCap\ Education}
\datedsubsection{\textbf{Carnegie Mellon University - School of Computer Science}}{Pittsburgh, PA}
\datedsubsection{\textit{M.S.} in Computer Vision | Current GPA: 4.22/4.3}{Dec. 2020}
\datedsubsection{\textbf{Beihang University - School of Computer Science and Engineering}}{Beijing, China}
\datedsubsection{\textit{B.Eng.} in Computer Science and Technology | GPA: 3.78/4, Graduation with Honors}{Jul. 2019}
\datedsubsection{\textbf{University of Toronto - Faculty of Applied Science and Engineering}}{Toronto, ON}
\datedsubsection{Scholarship-Funded Exchange Program | GPA: 3.88/4}{Dec. 2017}

\section{\faSuitcase\ Professional Experiences}

\datedsubsection{\textbf{Nuro Inc.}}{Mountain View, CA}
\datedsubsection{\textit{Senior Software Engineer, Perception} | Python, C++}{Jun. 2024 - Present}
Develop state-of-the-art perception models \& implement associated train/deploy infrasturcture
\begin{itemize}
  \item Developed large-scale E2E model infrastructure for joint perception-behavior training-deploy-eval workflow
  \item Performed in-training and post-training profiling to identify and resolve the latency/memory/tech-debt bottlenecks
  \item Reduced \textbf{10\%} latency by re-modeling the anchor-based design with heatmap-based design
  \item Re-developed detection model with basic tf/keras3/torch ops, meanwhile ensuring the whole model's numerical \& speed parity across backends, as part of the team effort to move away from tensorflow towards pytorch
\end{itemize}

\datedsubsection{\textit{Software Engineer, Perception} | Python, C++}{Jan. 2021 - Jun. 2024}
Develop modernized perception modeling infrastructure
\begin{itemize}
  \item Developed keras-based \href{https://github.com/open-mmlab/mmdetection3d}{\underline{MMDet}}-like framework that \textbf{unified the modeling workflow for perception team}
  \item Implemented unified TF-TRT custom operator API that supports automatic \href{https://docs.nvidia.com/deeplearning/tensorrt/operators/docs/PluginV2.html}{\underline{TensorRT}} export with \href{https://www.tensorflow.org/guide/create_op}{\underline{custom TF ops}}
  \item Re-implemented the entire camera-lidar 3D detection workflow (from data generation, model training to final deployment) with frameworks above to showcase its better performance (\textbf{significantly improved APs} with joint temporal training) and debuggability (ultimately getting \textbf{5x higher hours per interruption} without NaN / OOM, etc.)
\end{itemize}

\datedsubsection{\textbf{SenseTime Co., Ltd.}}{Beijing, China \& San Jose, CA}
\datedsubsection{\textit{Research Intern (San Jose Office)} | C++, Bash, Python}{May. 2020 - Aug. 2020}
Compression and quantization of neural networks for camera-related CV tasks on smartphones
\begin{itemize}
  \item \textbf{50\%} channel-pruning compression of CNN to obtain fine-grained quad bayer captured by \href{https://www.sony-semicon.co.jp/e/products/IS/mobile/2_2_ocl.html}{\underline{2x2 on-chip lens}}), enhanced the light-weight model (Python) for low-exposure frames with hard example fine-tuning
  \item \textbf{5x speedup} of CNN for bayer demosaicking on Xiaomi phone's raw data, achieved by mixed-bitwidth (16-bit activation and 8-bit weight) quantization-aware training (Python) with AIMET toolbox
  \item Developed the deployment pipeline for CNN models on smartphones (C++ and bash scripts), verified the model performance on the \textbf{DSP/CPU} of an Oppo Reno 2 and a google Pixel 3 with SNPE toolchain
\end{itemize}
\datedsubsection{\textit{Research Intern (Beijing Office)} | C, C++, Python}{Feb. 2018 - Jul. 2019}
Performance optimization and pipeline automation for deep learning frameworks and packages
\begin{itemize}
  \item Developed \href{https://github.com/ModelTC/NART}{\underline{pytorch-onnx-caffe conversion and profiling package}} supporting \textbf{all neural network layers}, effectively bridged the gap between research teams (training) and engineering teams (deployment) 
  \item Designed easy-to-use, modularized APIs that successfully worked with models within a wide variety such as pedestrian re-ID, face verification, car detection, etc. (number of users soon \textbf{exceeded 300} since first release in a month)
  \item Implemented novel neural network layers (time-shift operation, correlation convolution, etc.) in Caffe (C++) with research teams, \textbf{halved the train-test-deploy response cycle of any new model} 
  \item Developed inference framework (C) optimized for x86 processors with MKL-DNN, \textbf{2x speedup} compared to regular Caffe, used as deployment framework on development boards and light-weight chips
\end{itemize} 

\datedsubsection{\textbf{Robotics Institute, Carnegie Mellon University}}{Pittsburgh, PA}
\datedsubsection{\textit{Student Researcher, supervised by Prof. John Galeotti} | Python, C++}{Feb. 2020 - May. 2020}
Develop \href{https://mscvprojects.ri.cmu.edu/2020teamn/}{\underline{stateless relocalization module}} to fight the drifting problem in long-range UAV flights 
\begin{itemize}
  \item Implemented a fully convolutional neural network for scene coordinate regression, and applied \textbf{differentiable RANSAC with PnP algorithm} on scene coordinates for pose estimation
  \item Leveraged GPS and structure-from-known-motion with OpenMVG to obtain \textbf{high-quality ground truth} for training
  \item Averagely \textbf{<3m, <0.3° error} tested on 10-kilometer flight data, \textbf{<1m, <0.1° error} tested on 2-kilometer flight data
\end{itemize}

\datedsubsection{\textbf{FHL Vive Center for Enhanced Reality, UC Berkeley}}{Berkeley, CA}
\datedsubsection{\textit{Student Assistant III, supervised by Dr. Allen Yang} | Python, C++}{Jun. 2019 - Sept. 2019}
Develop and review new features for \href{https://vivecenter.berkeley.edu/research1/openark/}{\underline{OpenARK}}
\begin{itemize}
  \item Implemented \textbf{ICP algorithm} for SLAM module, stabilized the trajectory on texture-sparse frames
  \item Implemented a Caffe-based web demo for human face registration \& verification
\end{itemize}
\datedsubsection{\textit{Visiting Student Researcher, supervised by Dr. Allen Yang} | Python, C++}{Jul. 2018 - Oct. 2018}
Design a \href{https://arxiv.org/abs/1811.09938}{\underline{loop closure detection module}} and improve localization module for the lost track problem in VR/AR scenarios
\begin{itemize}
  \item Designed a \textbf{feature-pyramid siamese network} for loop closure detection w/ comparable performance to ORB-SLAM
  \item Synthesized a \textbf{large-scale} ($\sim$150,000 images) indoor environment dataset with Unity3D and SunCG for train \& test
\end{itemize}

\section{\faHeartO\ Awards and Certificates}
\datedline{An Image Retrieval System Based on Natural Language Captioning, \href{http://www.xjishu.com/zhuanli/55/201910738598_3.html}{\underline{CN Patent}}}{Aug. 2019}
{• Automatic image captioning upon uploading, used BLEU score as the key for retrieval, enabling descriptive search}\vspace{0.8mm}
\datedline{\textbf{$1^{st}$-place Winner} with \textyen 10,000 ($\sim$\$1,500) Prize, BeyondSoft Tech Challenge on Motion Evaluation}{Nov. 2018}
{•  Designed neural network to evaluate motion quality for athletes / rehabilitating patients on inertial data}\vspace{0.8mm}
\datedline{\textbf{National Scholarship} for Academic Excellency, Chinese Ministry of Education}{Nov. 2017}
{•  Top-level scholarship awarded nationally to recommended students for their academic excellency}{\vspace{0.8mm}}

\section{\faDesktop\ Skills}
Python, C, C++, Bash; Pytorch, Tensorflow, Keras, SciKits; TensorRT, ONNX, OpenCV, OpenMVG

\end{document}
